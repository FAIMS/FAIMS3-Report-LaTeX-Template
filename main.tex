%%%%%%%%%%%%%%%%%%%%%%%%%%%%%%%%%%%%%%%%%
% FAIMS3 Report
% LaTeX Template
% Version 1.0 (March 12, 2021)
%
% This template was created by:
% Brian Ballsun-Stanton and
% Vel (enquiries@latextypesetting.com)
%
%!TEX program = xelatex
% Note: this template must be compiled with XeLaTeX rather than PDFLaTeX
% due to the custom fonts used. The line above should ensure this happens
% automatically, but if it doesn't, your LaTeX editor should have a simple toggle
% to switch to using XeLaTeX.
%
%%%%%%%%%%%%%%%%%%%%%%%%%%%%%%%%%%%%%%%%%

%----------------------------------------------------------------------------------------
%	PACKAGES AND OTHER DOCUMENT CONFIGURATIONS
%----------------------------------------------------------------------------------------

\documentclass{faims3_report}

\bibliography{references.bib} % BibLaTeX bibliography file

%----------------------------------------------------------------------------------------
%	REPORT INFORMATION
%----------------------------------------------------------------------------------------

\reporttitle{FAIMS 3\\ Elaboration\\ Report}
\reportsubtitle{Design technologies for 2021-2022 the mobile app development cycle}
\reportfor{Technical Advisory Group \&\newline Steering Committee}
\reportlicense{Creative Commons 4.0 International - With Attribution}
\reportcontact{Brian Ballsun-Stanton \newline <brian@faims.edu.au>}
\reportversion{1.0.0-SC}
\reportdoi{10.5281/zenodo.4546985}
\reportauthors{ % Separate additional authors with \newline instead of \\
	Brian Ballsun-Stanton \newline
	Rini Angreani \newline
	Steve Cassidy \newline
	Simon O’Toole \newline
	Nuria Lorente \newline
	Elizabeth Mannering
}
\reportdate{\today}

%----------------------------------------------------------------------------------------

\begin{document}

%----------------------------------------------------------------------------------------
%	OPENING PAGES
%----------------------------------------------------------------------------------------

\maketitle % Output the title page, automatically populated using the report information specified above

\newpage

%------------------------------------------------

\textcolor{faimsblue}{Changelog}

\begin{tabular}{@{}l l p{.7\textwidth}}
	1.0.0-SC & 2021-02-25 & Typesetting in \LaTeX{}\\
	0.9.0-SC & 2021-02-18 & Report with comments to Shawn Ross for  advancement to steering committee. \\
	0.2.0-TAG & 2021-02-12 & Adding feedback from emails \\
	0.1.0-TAG & 2020-12-17 & Final typesetting and release to the Technical Advisory Group \\
	0.0.5 & 2020-12-16 & Edit by BBS \\
	0.0.4 & 2020-12-15 & Edit by Shawn Ross (versioned named before edit) \\
	0.0.3 & 2020-12-09 & First edit by BBS \\
	0.0.2 & 2020-12-03 & Release to dev team, Update QA work division wording, Updates on framework/platform choices, SC and BBS discussed frameworks and JSON Forms choices. \\
	0.0.1 & 2020-11-26 & Initial composition \\
\end{tabular}
\vfill
\copyrightnotice % Output the copyright notice defined in the class

\newpage

%----------------------------------------------------------------------------------------

\outputtoc % Output the table of contents, automatically populated from the various section commands

%----------------------------------------------------------------------------------------
%	REPORT BODY
%----------------------------------------------------------------------------------------

\chapter{Design objectives}

FAIMS3 is a ground-up rewrite of the
\href{https://github.com/FAIMS}{{FAIMS Mobile (v2.6)}} offline-capable,
geospatial, multimedia, field-data collection application
\autocite{Ballsun-Stanton2018-hq}. This rewrite is designed to be multi-platform, maintainable, and
to support data collection at a citizen-science scale. The code and
platform should last for at least five years\footnote{this is a footnote}, assuming regular
maintenance.

These are difficult objectives.

Our memorandum of understanding with the developers lists the following
intended capacities:

\begin{itemize}
\item Replicate all FAIMS v2.6 features that are currently used as part of
  three projects' research practice: CSIRO environmental geochemistry,
  LTU Mungo Lakes archaeology, and UNSW Oral History.
 
\item Allow self-service customisation and deployment via a web application
  without needing FAIMS team intervention for the vast majority of
  deployments.
 
\item Operate cross-platform, running the same code on Android, iOS, and
  `desktop'.
 
\item Allow data `round trip' to web and desktop applications. Round-trip,
  aspirationally, can be considered as data captured in-field, edited
  externally, then returned to the device in editable format.
 
\item Improved scalability and performance. Our target is ten times the
  number of records per deployment compared to v2.6 plus
  server-to-server synchronisation.
 
\end{itemize}

Pragmatically, we had to balance these objectives against available
skills, the costs of retraining/hiring, development budgets/timelines,
and future maintainability. During elaboration, we had several
objectives:

\begin{enumerate}
\item Try to quickly falsify technologies to rule-out unsuitable ones;
\item Build a team -- determine patterns of communication, management, and
  capabilities;
 
\item Explore where we needed to take on technical debt, what developmental
  affordances were supported by our choices, what external dependencies
  will be a worthwhile tradeoff between time saved versus external
  support risk, and where we will have to rebuild components because
  external support was insufficient;
 
\item Explore the technologies needed to support development and make a
  persuasive case for DevOps choices;
 
\item Produce elaboration outputs that indicate the feasibility of our
  desired development; and
 
\item Elicit feedback from the project community to provide an external
  perspective that can catch errors and suggest improvements.
 
\end{enumerate}

\chapter{Technologies}

\section{Programming language and
datastore}

Language-wise, our primary inquiry compared the modern Javascript app
development ecosystem versus promoted languages like Google's Dart.
Javascript has advanced significantly over the years, moving from a
browser-based web-scripting language to a modern, fully fledged
programming language capable of managing front-end experience and
back-end business logic. Dart, on the other hand, is a recently
designed, purpose-built language from Google that assessors found
persuasive in the FAIMS submission to the DataApp Challenge competition
as a `well-supported native-code compilation framework'
\autocite{Bureau_of_Reclamation2017-xl}. While Dart compiles to Javascript or native-code application,
none of the proposed programming team has experience with the language.

Initial inspection, unfortunately, was enough to discount Dart. While
Node.JS has excellent \emph{offline} NoSQL support with Apache's
PouchDB, and server-side synchronisation with CouchDB, no similarly
mature packages exist for Dart. We also found no persuasive mapping
packages. Dart is certainly a viable tool for standard web applications
built using a modern client-server or online serverless environment.
However, its design priorities and library support do not emphasise
multi-OS, geospatial, offline capabilities. Furthermore, Dart's library
environment is less mature than Node.JS. This lack-of-maturity for
offline and geospatial support, combined with Google's proclivity to
abandon underperforming projects, argued against Dart.

We also had to decide between writing native applications versus making
a pure Javascript webapp. Native applications in Java, Objective-C, and
a language appropriate for desktop provide lower-level access to
operating systems, sensors, and device capabilities in exchange for a
tripled programming load. At the other end of the spectrum, we could
pursue a pure-browser progressive web application approach, offering a
seamless multi-platform experience in the browser. We chose neither of
these routes. On one hand, we could not deploy a pure progressive web
application that runs only in a browser since we had to access device
capabilities that are carefully and appropriately isolated from the
browser. On the other, we chose not to deploy a purely native
application due to the expense of development without a compelling need
for the performance gains offered by low-level operating system access.

We chose the middle route: wrapping a progressive Javascript
`\href{https://en.wikipedia.org/wiki/Single-page_application}{{single
page application}}' in native code. That way, we can `deploy an app'
(native code) to all platforms (AppStore, Google Play, Microsoft Store)
but written in Javascript and shipped with enough webbrowser-like code
so that it acts and appears as a normal application on someone's mobile
or desktop device. This approach, deploying `native progressive webapps'
is an extremely common programming practice for those projects needing
to deliver a multiplatform experience without the ability to support
three or more distinct programming teams.

\subsection{Programming language}

We will use ECMAScript 6, known as Javascript, with JSX code extensions.
This choice was primarily motivated by the React framework and ecosystem
-- though we are still investigating the need for a framework like
React at all. We will be using Node.JS v14 for Alpha development as it
is the \href{https://nodejs.org/en/about/releases/}{{current LTS
release}}. It will reach end-of-life on 2023-04-30. If possible, we will
try a version upgrade to v16 as early as possible to extend EOL to 2024.
We recognise that these short `long-term service' releases will impose
substantial maintenance -- but as we plan for a yearly maintenance
cycle to allow testing and ensure support for major Android and Apple
releases, bumping and testing the node version will be part of that
plan. Furthermore, we anticipate a need for more frequent security
releases due to the enormous complexity of the Node.JS package
ecosystem.

All base components of FAIMS3 developed by us will be written in
Javascript: the application, the module designer, and whatever
supporting server infrastructure is required. There may also be a need
for OS specific plugins to support specific functionality, though we
will try to avoid this exigency whenever possible.

\subsection{Datastore}

We have chosen to use the CouchDB document store as our fundamental
database. It is a robust multi-master document store designed to deal
with intermittent connectivity and opportunistic synchronisation. It is
ACID-compliant and supports integral document versioning
\href{https://docs.couchdb.org/en/stable/api/database/misc.html\#db-revs-diff}{{implying
support}} for a
\href{https://dba.stackexchange.com/a/248675}{{version-history}} similar
to that of FAIMS2.6. It also has a stronger focus on
\href{https://severalnines.com/database-blog/battle-nosql-databases-comparing-mongodb-and-couchdb}{{mobile
application support}} via \href{https://pouchdb.com/}{{PouchDB}}.
Finally, the Apache License used for
\href{https://db-engines.com/en/system/CouchDB\%3BMongoDB\%3BPouchDB}{{CouchDB
and PouchDB}} is more permissive than the viral AGPL. We have therefore
chosen the robust and mature pair of CouchDB 3 / PouchDB 7 as the FAIMS3
datastore. Elaboration experiments with these DBMSes in React and
React-Native have proven satisfactory. We hope to be using GeoCouch as a
spatial index addon for CouchDB and GeoPouch as its local complement but
both GeoCouch and GeoPouch are aging projects that do not have current
updates. While there is legitimate concern around GIS performance, we
have not investigated implementation options for the specific geospatial
functionality and tradeoffs needed for our target users and their
modules.


\section{Framework, essential packages, and
toolchain}

\subsection{Framework}

We are not yet convinced that we want to take on the technical debt of
the React framework. However, we have investigated React and React
Native, and we undertook preliminary investigations of Ionic and Vue. As
a result, if we do use a framework, it will be
\href{https://reactjs.org/docs/faq-versioning.html}{{React 17}}. React
17 supports gradual upgrades and, considering the current developer-base
of the framework, it will still be in use in 2025.

The choice of React is driven by the experience the team has with it and
the lack of clear differentiators with Vue. We did not investigate
Angular due to the expert advice of our programmer who is involved in a
complex and entailed migration away from Angular.

The main alternative to using React would be to use a collection of more
limited frameworks for specific tasks such as building web components,
producing page layout, etc. The main advantage of this approach would be
reduction of overall project dependencies, but we currently believe that
the value of React outweighs its concomitant dependency risks.

\href{https://github.com/FAIMS/faims3reactnative}{{One of our
elaboration exercises}} was exploring React Native. React Native did not
present the same capabilities as React in terms of its supported
packages and is not yet at a stable release. Due to its relative
immaturity and lack of compelling advantages over React, we chose not to
explore React Native further.

We also chose not to explore the Ionic framework. While it does have
some useful upgrades for Capacitor, discussed in the next section, the
features offered in either its open source or enterprise versions were
not required during elaboration. During development, however, if one of
those Ionic plugins for Capacitor are needed, we would need to make a
judgement call about including the whole Ionic framework.

\subsection{Packages and native
runtimes}

\subsubsection{Compiling to Native
runtimes}

Our cross-platform commitment requires systems able to compile to Apple
and Android native runtimes, as well as building to Electron's native
desktop wrappers and a Progressive Web Application for pure web browser
support. We investigated Apache Cordova, Ionic's Capacitor, and
Facebook's React Native for this purpose. React Native did not offer
compelling
\href{https://ionicframework.com/blog/ask-a-lead-dev-react-native-or-ionic/}{{advantages
for our use-case}}, and increased development friction. React Native's
plugin ecosystem also appears weaker compared to the first-class support
from Ionic. While Cordova \emph{nee} PhoneGap was initially our
preferred native runtime, it does not offer the same access to web APIs
and
\href{https://ionicframework.com/resources/articles/capacitor-vs-cordova-modern-hybrid-app-development}{{native
platform capabilities}}.

Ionic's open source offering in this space is Capacitor. Capacitor,
billed as the successor to Cordova, compiles to a
\href{https://ionicframework.com/docs/reference/glossary}{{Native
Progressive Web App}}, giving us the flexibility to include native code
for OS-specific features. Capacitor offers critical capacities required
for feature delivery and performance. Capacitor compiles its Javascript
into platform-specific IDE builds which allow us to `add custom native
code...without having to build a new plugin for it [and]
\href{https://capacitorjs.com/docs/cordova}{{provides better app
maintainability as new mobile operating system versions are released}}'.
This support for mobile operating system upgrade and migration is
critical to FAIMS3's long-term sustainability and is a major benefit
over Cordova for this project.

We investigated Ionic Enterprise due to their claims of `enterprise
security updates' and its Capacitor enterprise plugins. We have,
however, seen no compelling features in its enterprise support. If we
decide that JSON Forms, when implemented in detail, do not provide the
schema support we require, we will investigate Ionic again, since it
offers some
\href{https://eclipsesource.com/blogs/2018/12/21/json-forms-goes-mobile-with-ionic/}{{additional
functionality in that regard}}.

\subsubsection{JSON Forms}

We are provisionally satisfied with the functionality of
\href{https://jsonforms.io/}{{JSON Forms}}, which renders a JSON Schema
as a dynamic form. Although we have some concerns about its capacity to
accommodate the complexity we will ask of the system, we have not found
a persuasive alternative for dynamic form rendering based on a JSON
schema. If JSON Forms fails, we will need to contribute to the package,
fork it, or build our own. This will be a significant but necessary
expense, but one that would result in downscoping other features.

The main shortcoming of JSON Forms arises from the JSON Schema standard
itself, which does not allow sub-typing (defining types as
specialisations of other types). This lack of subtypes makes it hard to
extend the basic set of types to cover new kinds of input fields. For
example, we would like to have a library of predefined types like `GPS
Location' that could be used in any application. Two things stop us.
First, we cannot build a 'standard library' of these custom subtypes in
the style of the FAIMS v2.6 autogenerator. Each JSON Schema would need
to be entirely self-contained. Secondly, even if we declared these
subtypes in a schema, the declaration is not of an abstract type but of
a full form element. Having two or more GPS Locations in one form would
either require them to have the same `prompt' (form label) or be defined
as separate types (top-left-GPS-location, secondary-GPS-location, etc).
This constraint on custom fields means that we will need to extend the
JSON Schema to allow us to write appropriate specifications. Such
development would also prevent use of any existing JSON Form library in
our application.

The JSON Schema is basically sound, but we need to build our own version
that meets project requirements. The architecture of the JSON Forms
library is quite complex, as it handles multiple host frameworks and
multiple output rendering options. We can learn from and build upon the
existing library but we will need our own implementation. A major effort
of alpha development will be to design and build a sub-system that can
instantiate appropriate forms from a JSON representation.

\subsubsection{Leaflet}

The other major avenue of investigation during elaboration was mapping /
mobile GIS support. We currently favour an approach using Leaflet, a
Javascript tile-render. Leaflet supports many of the features
researchers used in FAIMS v2.x -- raster maps, vector maps, dynamic
points, vectors, and
\href{https://leafletjs.com/\#features}{{rudimentary vector styling}}.
It also utilises hardware acceleration on mobile devices, which should
result in a higher performance. We are investigating multiple offline
modes for Leaflet, either with a global caching functionality or
something like
\href{https://github.com/allartk/leaflet.offline}{{Leaflet Offline}},
another tradeoff between development time and external dependencies.
Since Leaflet is an open-source javascript project, however, it does not
present the same external library trap as Nutiteq (a proprietary mobile
GIS) did for FAIMS v2.x. Nutiteq locked us into a specific Android API
and Sqlite version and then stopped supporting their code. Because
Leaflet does not dictate our datastor or ship with native code that we
have to incorporate into our application, it is much easier to switch it
out later.

Geospatial storage has advanced significantly since our last elaboration
in 2013 (when the only viable option was Spatialite extensions to SQLite
rendered via a mobile GIS). We will store GeoJSON in our CouchDB
document store. Hopefully, as discussed earlier, using the
\href{https://github.com/couchbase/geocouch}{{GeoCouch spatial
extension}} to allow for spatial queries to reduce rendering complexity.
We do not anticipate building significant GIS query capabilities into
our fundamental datastore as those expensive features were underutilised
by clients in FAIMS v1.0-2.6. Instead, we will rely on building that
functionality into plugins when explicitly required (and funded) by
clients. A more modular design will also allow us to replace components
as they reach end-of-life without impacting the rest of the application.

However, other members of the FAIMS Leadership team have experience with
Leaflet and geojson and note that both have a ceiling when it comes to
fast display of complex and large (10000+) geospatial datasets. We do
not plan to implement significant GIS functionality in the application,
but being able to visualise collected data is an essential requirement.
Therefore, it is important that users be able to visualise all the
points they have collected on an open street map or satellite view as a
minimum basis for field usability. While some projects will need to
collect more sophisticated GeoJSON objects (lines, linestrings, and
polygons), we anticipate that support for the collection of these
objects will happen in a later development cycle -- not because the
library does not support it, but because the user interface needed to
specify and interact with complex geometries will consume significant
development time and is not part of the three modules we are targeting
as part of this development cycle. However, all of the fundamental data
structures support a more sophisticated mobile GIS -- which we develop
with successive rounds of funding.

However, more projects will need to visualise external vector data. To
increase GIS vector display performance, we hope to use
\href{https://github.com/Leaflet/Leaflet.VectorGrid}{{Leaflet.VectorGrid}}
or
\href{https://leafletjs.com/plugins.html\#vector-tiles}{{Leaflet.VectorTileLayer}}
as a way of precomputing and rendering these vector tiles. However, we
did not have enough built during elaboration to justify significant
exploration of GIS capabilities in this phase. As a development
priority, however, we will focus more on GIS performance than features.

\subsubsection{Dynamic Plugin
Architecture}

Because we will be building a `progressive native web application'
written in Javascript, we will rely heavily on a dynamic plugin
architecture. As many aspects of the application as possible will ship
as plugins. This approach provides a modular structure that avoids
vendor lock-in. It also supports custom functionality required by
clients without recompiling the application. We hope to be able to have
modules load custom, module-specific, plugins downloaded from a central
server. For example, instead of compiling Zebra bluetooth printer
support into the core FAIMS application, we hope to load the necessary
bluetooth-serial plugins dynamically when users request the CSIRO
module. This will improve performance for modules that do not require
the additional functionality and allow us to more easily customise
specific user experiences. Our intended plugin architecture will support
Javascript and webassembly.

We have not significantly elaborated on our plugin architecture. While
there are indications that
\href{https://developer.android.com/guide/app-bundle/play-feature-delivery}{{Android}}
and
\href{https://theswiftdev.com/building-and-loading-dynamic-libraries-at-runtime-in-swift/}{{iOS}}
support dynamic loading, there is no evidence that Capacitor integrates
into these specific native code features. We hope to also support Java
(Android) and Swift (iOS) dynamic plugins, but that hope is not
currently explored in the literature.

Where possible, we will strongly prefer Javascript plugins for
maintainability, although some tension exists between Javascript and
native binding-reliant elements. If we use dynamic, native-code plugins,
they will have to be
\href{https://www.joshmorony.com/creating-a-local-capacitor-plugin-to-access-native-functionality-ios-swift/}{{transpiled
into objective C and Java} {as needed}}. Only in the last resort will we
write new Objective C or Java. Managing these development time tradeoffs
will be one of the major factors of plugin versus core functionality.


\section{Application Programming Interfaces
(APIs)}

One of the limiting factors of FAIMS 2.6 was the lack of
interoperability with external programs, requiring dedicated on-server
`exporters' to transform module data and produce files compatible with
spreadsheets, ArcGIS, and Google Earth. FAIMS 3 will instead support a
modern API for data exchange and other interactions.

\subsection{Datastore access}

A highly normalised, append-only relational database was required in
FAIMS v1.0-2.6 to support profound customisation and robust versioning,
but it made data export difficult and data import nearly impossible. In
FAIMS 3.0, we will open our data to external services. We plan to use an
unmodified CouchDB instance as our server-side datastore. CouchDB uses
well documented RESTful APIs to create, read, update, and delete data.
Therefore, most external interactions with our server do not involve the
mobile application; users only need to know how we store record
`documents' in the database.

\subsection{Application API}

The FAIMS 3 application, however, should function both as a data
collection app and a 'server,' handling module coordination and data
synchronization within the same codebase. As a result, individual
instances of the app will need to be able to communicate to each other:
changes in data, modifications in vocabularies, and sometimes even
acting as a server, given an instance of the app running inside a cloud
virtual machine.

Thus, the FAIMS 3 app will support an API allowing module creation and
access alongside data creation, reading, updating, and deleting. While
access to the CouchDB datastore will be permitted, some partners may
prefer to engage with a single, consistent record-level API rather than
implementing record-parsing inside of their systems.

\subsection{`Round trip' functionality
}

At present, there are too many dependencies in any `round trip' between
FAIMS and data editing or analysis software to decide on technology.
Interaction with the Application API will likely be the basis of any
data `round trip' with properly configured external programs. Our
fallback is provision of a
\href{https://www.npmjs.com/package/editable-table}{{tabular
multi-record editable data view}} with interactive editing (i.e.,
something like a spreadsheet).
\href{https://www.npmjs.com/package/xlsx}{{Libraries that can interact
with `excel workbooks'}} suggest the viability of this approach.
Potential integration with desktop applications and existing online
services requires further exploration.

There are also indications that the
\href{https://www.ogc.org/standards/wfs}{{Web Feature Service
specification}} offers a way to expose geospatial vector features in an
\emph{editable} fashion to GIS software. While we have not investigated
how to engage with this specification, we are optimistic that some sort
of GIS round trip is theoretically supported by external vendors.

\subsection{Data exporters}

An exporter is a data manipulation sequence, so it is either implied in
the round-trip functionality, or is well demonstrated by FAIMS 2.6
exporters.

\section{DevOps, QA, CI/CD}

\subsection{DevOps Philosophy}

Tooling and infrastructure choices have been made to minimise system
administration and server requirements. As such, exploiting open-source
development incentives offered by GitHub, Atlassian, and BrowserStack
were fundamental to our approach. GitHub's offerings for
\href{https://github.com/account/organizations/new?plan=free\&ref_cta=Sign\%2520up\%2520your\%2520team\&ref_loc=changing\%2520the\%2520world\&ref_page=\%2Fpricing\&source=pricing-open-source}{{Open
Source Teams}}, \href{https://education.github.com}{{Education}}, and
public repositories allow us to run our CI/CD pipeline for free along
with maintaining a robust set of version controlled repositories.
Atlassian's
\href{https://www.atlassian.com/software/views/open-source-license-request}{{Open
Source Project License}} allows us to use a professional Jira and
Confluence instance for free, hosted by Atlassian.
\href{https://www.browserstack.com/open-source}{{BrowserStack for Open
Source}} allows us to run device level end to end and integration tests
on real and virtual devices. Utilising these software-as-a-service
offerings allows us to demonstrate a software delivery pipeline that can
be maintained long term by 0.2-0.4 FTE.

The next major constraint was to automate testing to the extent
possible. One flaw with FAIMS v2.x was that updates required more than a
week of manual testing. This manual workload rendered regression testing
expensive and performance testing or end-to-end tests impossible. As
FAIMS3 is designed to be self-service -- from module generation to
data export -- automated end-to-end testing was a fundamental
requirement.

Testing responsibilities will be divided between the teams. Our
programmers will be responsible for writing unit tests in Jest. These
Javascript unit tests will test low-level code functionality that does
not interact with the user interface. Our external QA team will be
responsible for integration and end to end testing. This higher level
testing will ensure that all the components work together, that the user
interface is tested, and will make sure that features as specified in
user stories are exercised. CSIRO will also be responsible for
double-checking our code documentation, to make sure it is informative
to someone without immediate access to the developers. This approach
will create a project that will ensure a level of professionalism and
robustness that allows for deployment and operation at scale -- and
may attract external developer contributions. This design decision,
however, reduces feature development due to the higher quality assurance
demands placed on each feature. We believe that the tradeoff is worth it
for software being developed for a five-year lifespan.

\subsection{Quality assurance}

Unit testing will be written in Jest and run as part of a
\href{https://github.com/FAIMS/FAIMS3-Elaboration/blob/master/.github/workflows/node.js.yml}{{Github
Actions pipeline}}. Unit tests will be evaluated as a pre-commit hook,
using GitHub actions on commit and pull requests. Unit tests will also
be discussed as part of sprint planning and demonstrated during sprint
demos. One measure of quality assurance is code coverage, how many lines
of code are `exercised' by Unit Tests. By measuring
\href{https://jestjs.io/docs/en/cli.html\#--coverageboolean}{{code
coverage of unit testing}} as one of our development goals, we can
hopefully make sure that we have a robust and comprehensive testing
suite at all levels -- suitable for five or more years of maintenance.
Unit testing, however, does not include UI testing, integration testing,
regression testing, or end-to-end testing. CSIRO will be responsible for
developing the integration tests that will provide assurance that all
components fit together and deliver our intended outcome. One design
goal is to ensure that these integrated tests can also function as part
of an end-to-end testing regimen so that the entire application can be
exercised nightly. Unfortunately, test development is time-expensive, so
we anticipate some features to require manual regression testing despite
the planned end-to-end framework. Features that depend on external
devices or sensors are especially likely to require manual testing.
Nevertheless, if we can minimise manual testing when shipping a new
version, we can release more often and with more confidence.

\subsection{CI/CD Pipeline}

Github Actions will provide our fundamental continuous integration /
continuous delivery pipeline, compiling on commit, running Jest Unit
Tests, using Capacitor, Fastlane, and Electron to build a PWA, desktop
installers, and producing signed code ready for deployment to Google
Play and the Apple AppStore. These binaries will then automatically
upload to BrowserStack App Automate for integration tests and end-to-end
testing.

We are using the generous Open Source options for both GitHub and
Browserstack for this purpose, which should provide sufficient build
minutes to maintain a good testing pipeline at a very low cost.

\chapter{Non-elaborated
functionality}

\section{Anticipated but untested functionality
}

We did not test extended Capacitor functionality because we needed to
prioritise basic data structures, language features, and
\href{https://capacitorjs.com/docs/plugins/community}{{map-showing
functionality}}. Documentation, however, indicates that the following
required functionality exists as Capacitor plugins:

\begin{itemize}
\item Basic geolocation:
  \href{https://capacitorjs.com/docs/apis/geolocation}{{https://capacitorjs.com/docs/apis/geolocation}}
 
\item Basic file access:
  \href{https://capacitorjs.com/docs/apis/filesystem}{{https://capacitorjs.com/docs/apis/filesystem}}
 
\item Basic camera:
  \href{https://capacitorjs.com/docs/apis/camera}{{https://capacitorjs.com/docs/apis/camera}}
 
\item Save/play video:
  \href{https://github.com/capacitor-community/media}{{https://github.com/capacitor-community/media}}
 
\end{itemize}

If we choose to include
\href{https://www.smashingmagazine.com/2019/08/building-mobile-apps-ionic-react/}{{Ionic
as well as React}} in our dependencies, we would increase our dependence
on external vendors and libraries they choose to import, but we would
gain additional features for little development time:

\begin{itemize}
\item A barcode scanner (CSIRO):
  \href{https://ionicframework.com/docs/native/barcode-scanner}{{https://ionicframework.com/docs/native/barcode-scanner}}
 
\item Raw Bluetooth serial connections for mobile printing (CSIRO):
  \href{https://ionicframework.com/docs/native/bluetooth-serial}{{https://ionicframework.com/docs/native/bluetooth-serial}}
 
\item A document scanner (Oral history for PICF forms):
  \href{https://ionicframework.com/docs/native/document-scanner}{{https://ionicframework.com/docs/native/document-scanner}}
 
\item Audio capture (Oral history):
  \href{https://ionicframework.com/docs/native/media-capture}{{https://ionicframework.com/docs/native/media-capture}}
 
\end{itemize}

External GPS support (a common requirement) would require us to fork,
adapt, and support a Cordova plugin and Javascript library:

\begin{itemize}
\item \href{https://github.com/heigeo/cordova-plugin-bluetooth-geolocation}{{https://github.com/heigeo/cordova-plugin-bluetooth-geolocation}}
 
\item \href{https://github.com/infusion/GPS.js/}{{https://github.com/infusion/GPS.js/}}
 
\end{itemize}

Basic GIS analytics (Lake Mungo for finding grid squares) requires:

\begin{itemize}
\item \href{https://turfjs.org/docs/\#intersect}{{https://turfjs.org/docs/\#intersect}}
 
\end{itemize}

\section{Functionality not yet
elaborated}

FAIMS 3 has many planned features that do not exist as libraries to
include within our planned Javascript framework, or that depend so
strongly on infrastructure and architecture decisions that anticipating
a solution now would be premature. Functionalities that we anticipate
but have not yet elaborated include:

\begin{itemize}
\item Offline Map loading and serving. One option is utilising something
  like the \href{https://www.npmjs.com/package/shapefile}{{Shapefile
  Package}} during module generation to convert vectors into GeoJSON,
  plus an \href{https://www.npmjs.com/package/mapeo-server}{{offline
  tile server}}. Offline maps, GIS, and map generation is a part of
  FAIMS 3 that could consume extensive resources and it must be
  carefully managed. Both CSIRO and Lake Mungo require the display and
  management of vector and raster layers, but we have not yet explored
  trade-offs between options. Leaflet has demonstrated that it can
  render offline maps and points, so our fallback will be to run a
  `normal' map tile server and then allow caching of maps on the device.
 
\item UI elements (especially `dynamic' UI). We have not yet explored
  interface elements like `tabs' and `tab groups', including the ability
  to dynamically alter input fields based on user interactions, as used
  in FAIMS v2.x. Dynamic manipulation of HTML is an extremely mature
  technology.
 
\item GUI module generator -- depending on complexity, we will need to
  decide between schematic representation (lists of elements) versus a
  more WYSIWYG editor. However, developing a standard web application
  which does not require offline support which can generate a module
  specification through some means is not of concern.
 
\item External data importers are not currently planned during this
  development cycle, but may be implied by the round-trip functionality
  or, at minimum, developed as a later plug-in (accommodated by the
  architecture discussed above).
 
\item On-load \href{https://react.i18next.com/}{{Internationalisation}} is
  well demonstrated in FAIMS 2.6.
 
\end{itemize}

OSGeo integration. Use of GeoCouch, GeoTools, and GeoServer have
implications for map-serving and roundtrip capabilities.

\chapter{Team Composition and Development
Plans}

FAIMS 3 will be developed by three groups: FAIMS leadership, AAO, and
CSIRO.

\begin{itemize}
\item FAIMS leadership will be responsible for planning and execution --
  development planning before each development period, assessing
  development via a demo after each period, providing subject matter
  expertise, and writing user stories and appropriate acceptance
  criteria for the stories. Leadership is also responsible for DevOps
  and ultimately ensuring that all the systems work.
 
\item AAO will be responsible for primary development -- implementing
  agreed user stories each development period, performing demos on a
  regular interval (every two weeks), writing unit tests, and
  documenting their code so that external developers can contribute.
 
\item CSIRO will be responsible for QA -- writing integration and
  end-to-end tests against AAO's user stories and documentation. By
  shipping code across the country, requiring that a second team
  understand, execute, and test the code, we hope to avoid testing
  situations where the testers lean across the office corridor and ask
  the developers for help rather than documenting usability or
  documentation deficiencies.
 
\end{itemize}

We will (mostly) follow an agile development methodology informed by the
Rational Unified Process. In short, each 'sprint' will be preceded by a
development planning meeting where user stories are allocated and a
general work consensus is achieved, ensuring that the plans are
reasonable, with time allocated for testing, documentation, and bug
fixes. No formal point/time allocation per story will be part of this
process, only informal estimates to better calibrate expectations to
development pace. After each two-week work-block, each story worked on
will be debriefed: working tests will be demonstrated and problems will
be discussed as a way of improving future planning. Where features are
interesting to the larger community, FAIMS Leadership will turn the
demonstration/retrospective into a
\href{https://factorio.com/blog/}{{regular blog post}}.

Everyone involved is very aware that agile development means that we
spend money for a specified period of time, rather than agree on a
formal waterfall Software Development Life Cycle. Development thus
intends to achieve the design objectives stated above, though some
descoping will likely be required, as we are prioritising quality
assurance over feature development.

Each external milestone will be preceded by User Acceptance Tests. These
tests will be written with input from FAIMS leadership, AAO, and CSIRO
so that external assessors can work through the features that best
demonstrate development progress. Only when everyone is satisfied that
the tests have been passed will software be released to external users.

\chapter{Future Decisions}

\section{Dependency hell is a
place}

The fundamental thing we need to decide is how much external code to use
in the FAIMS3 `core'. Inevitable external code dependencies exist in
Bluetooth, data transport, and GIS libraries, but keeping the core free
of excessive Javascript packages will increase maintainability at the
cost of longer development time.

We could, however, decide to commit to React + Ionic (Community or
Enterprise) + Capacitor + JSON Forms + `all of the plugins'. This route
gets us many more features at the cost of being beholden to many
different communities for maintaining their pieces of code.

We will almost certainly thread a needle between these extremes -- but
wish to acknowledge the tradeoff explicitly. We need to deliver the
required features, but avoid being trapped by an unsupported package
that exposes us to a security vulnerability or
\href{https://www.theregister.com/2016/03/23/npm_left_pad_chaos/}{{major
component failure}}.

\section{GIS Features}

We need to determine what subset of GIS functionality is actually used
in the field by our users (as opposed to what users claim they need),
and what external connections we need to support desktop GIS
applications. We developed many expensive GIS features in FAIMS 1.0-2.6
at the urging of researchers that were never used in the field, an
expensive error we must not repeat.

\section{The Round Trip Problem}

We need to figure out what users require from `data round trips' and how
much can be delivered in-application versus exporting bundles of CSVs.
There are some fundamental issues with data round-tripping:
denormalisation, preserving internal identifier row-associations,
preserving vocabulary keys, and preserving GIS-appropriate associations.
We also need to be somewhat program-agnostic, so that users can bring
the programs they are used to working with to the field camp for evening
analysis. There is also a risk of building very technical bridges to
specific programs that will end up being ignored by users. Determining
appropriate compromises to allow for bulk data review and editing --
tabular, multimedia, and geospatial -- will be a significant challenge
determined by our data structure and development choices.

\section{External Partner Support}

FAIMS 3.0 must interact with some of our external partners'
infrastructure: OpenContext, tDAR, and Cloudstor. At present we plan to
deploy well documented APIs to access the FAIMS 3.0 application and data
in the CouchDB instance. We also plan to add API function calls as
requested by our partners to support a more interactive export style.
One example could be that a user from one of these external services
enters a FAIMS 3.0 application URI and authenticates appropriately --
and then the service can prepare their data for export and ingest into
their own system. Planning where the bulk of data preparation will
occur, however, requires a better sense of our final architecture and
conversations with our partners. We plan to begin an external service
elaboration when FAIMS 3.0 enters its beta version User Acceptance
Tests. At that stage we will explore scenarios with each partner to
accommodate user needs.

\printbibliography


\chapter{Technical Advisory Group
Comments and Responses}

From Kate Robertson, email:
\begin{quote}
I've had a look at the document, no comments/changes etc from me, looks
good- thorough and well explained.
\end{quote}

From Richard Adams, email:
\begin{quote}
Just a few specific points -

1) We've found react.js to be excellent in our project and we are
committing to refactoring all our UI code to use react. It has such a
vast usage that it's extremely unlikely to get abandoned over the next 5
years or so. The problems it solves such as easy reuse of components
enabling a more standard UI has made a vast difference to the appearance
and behaviour of our application, and even if initially slower to get
started with, bears fruit over time as new pages can often be composed
of existing components

2) The API - not sure if this API is an `internal' API for instances of
the app to communicate and exchange data with each other/ or the local
server, or a public API for 3rd party software to access the data. If
the latter, have you considered having an intermediate data layer so
that the API isn't tied to the underlying data structures in CouchDB,
and the API and the internal data structures can evolve independently of
each other? This would help with maintaining stability of the API for
3rd-party developers

3) Testing and QA

 Always a thorny issue. For RSpace, we initially wrote manual test
  scripts for all features, then employed someone to automate them as
  much as possible using Selenium. This is an enormous effort to
  maintain; the tests can be fragile and with over 300 of them the false
  failure rate is difficult to keep to acceptable levels. For a new
  module we are instead doing more exploratory testing as means to
  identify defects; performing manual, unscripted testing for fixed
  periods of time. If the defect rate increases, we do more testing; as
  the defect rate falls we do less testing. Expected behaviour is
  defined in the use-cases and requirement specs; we use decision tables
  and state matrices to formalise key behaviours and states. We found
  this cuts down a lot of duplication of writing out requirements, then
  writing a largely similar testing document.
 
The aim is to do `just enough testing but no more' ie- how little
  acceptance testing can we do, before we get an unacceptable defect
  rate? So far, results are encouraging. Also, for software designed to
  be used by humans, manual testing by humans is essential, and actually
  beneficial due to insights into user-experience that automated testing
  would ignore.
 
Having said that, there is some scope perhaps for a small number of
  automated integration `smoke tests' that can detect egregious defects
  on a nightly build. We recently evaluated Cypress.io and liked it as a
  possible successor to Selenium for test automation; I understand they
  are developing a component-testing framework as well.
 
Code coverage is OK for catching gaping holes in test coverage - for
  example finding whole files or components; but a coverage tool has to
  be able to indicate branches and statements covered as well as just
  lines, in order to make sure that infrequently-called blocks are also
  tested ( for example exception/failure handling code).
 
Are you planning to do code-reviews as part of your Git workflow, e.g
  before merging a feature branch to mainline? We have found these
  useful; people really improve their code if they know their colleagues
  are going to comment on it; and it also implicitly spreads knowledge
  around the team and helps conventions to become established. Obviously
  this depends on the size of the team, but even in our tiny team of 2
  front-end and 2 backend devs, code quality and consistency has
  markedly improved since we started requiring manual code review before
  merge.
 
You mention that performance testing will be possible with this new
  process but it's not elaborated further - are some performance goals
  going to be established? I expect a mobile app in the field, efficient
  power usage is essential to maximise battery life, is that a design
  goal at all?
 

4) Devops

This all sounds great. Automation of the build and deployment as much as
possible is definitely the way to go. As opposed to manual testing,
there are almost no benefits to be obtained from manual builds.

5) Project schedule

Re the agile approach - are these going to be internal sprints, doing a
number of iterations before public release, or have you considered doing
very early alpha releases and inviting keen users to try out the
software, even before reaching Minimum Viable Product stage? We've been
trying this approach for a new product. It takes some time to coordinate
the volunteers, manage expectations and organise the feedback but we've
found several benefits - important features that we had missed out on
were detected; user experience on a variety of devices; public
visibility that the project is making progress.

6) General questions

How does data collected by FAIM{[}S{]} get linked to non-field related
data for a project? For example, for an archaeological project, there
will be the field data and also perhaps laboratory data examining the
artefacts - would all this be put in FAIM{[}S{]} database or would a
research team use other data management software too? My interest here
is if an ELN could be used to tie everything together.

7) This sounds like a fascinating project. I'd be particularly keen on
helping out or testing any external APIs to work with
FAIM{[}S{]}-acquired data, and of course happy to follow up with more
information on any of the testing/ PM ideas I mentioned above.
\end{quote}

Response to Richard Adams
\begin{quote}
\begin{enumerate}[itemsep=1em]
\item Thanks for the advice regarding React. We plan to have some sort of
  consistent UI framework. Steve has explored a version of the
  elaboration prototype without react and has reported some significant
  size/complexity reductions. However this will be dictated by what
  specific components React offers us. The dependencies required by react
  are a non-trivial cost.
 
\item Right now the plan for the API is to be external-facing, as we
  are trying to avoid needing to code and maintain a "server" on top of
  everything that FAIMS 3 will already be. Some light data abstraction
  layers, or at least data-format-consistency is desirable and
  will likely be requested by our external partners. How we're going to
  achieve that data consistency is not yet determined.
 
\item We unfortunately have had the opposite problem with testing and QA. 
  My goal for this testing regimen is to avoid after-hours, emergency tech 
  support over poor internet connections to projects in the field. Again. 
  The risks of data-loss and the costs of patching are much higher for an 
  offline system like ours. Beyond that, human costs while testing are one 
  of the reasons why FAIMS 2 aged so poorly, as we were not able to sustain 
  maintenance development with the week-plus regression tests needed. 
  Conversely, we recognise the costs inherent to the proposed QA approach. 
  My goal is to automate as much QA as we can afford -- to minimise data-loss and
  emergency tech support to offline (or marginally online) projects in remote areas. 
  This challenge is somewhat different from supporting users who have normal access 
  to their servers and infrastructure.
 

  \begin{enumerate}
  \item Yes, we also desire code-reviews. The specific pragmatics of the
    situation have yet to resolve.
   
  \item The performance goals are mostly in relation to data capacity and
    access. FAIMS 2, due to the append-only datastore, had issues with
    high numbers of complex records. Automated performance testing (which we hope 
    to implement again) allowed us to estimate when performance would drop off 
    (i.e., at how many records) and plan projects appropriately. We hope the use 
    of a noSQL database and other changes 
    will mitigate these performance issues, but want to make sure, since we can't 
    rely on online storage or processing during data collection and are constrained by device capability.
    In our requirements for FAIMS 3, we target `ten times the number of records than in FAIMS 2' 
    -- tens to hundreds of thousands of rows. 
    
    For mobile battery, the primary consumer of the battery 
    is the screen (some optimisation may be possible here, but is probably outside
    of our budget). With the partial exception of tracklogs (i.e., constant GPS use, 
    especially under canopy) on some devices, we had very few problems with battery 
    life in FAIMS 2 and do not expect any to arise (user testing should reveal any such
    problems early on). 
   
  \end{enumerate}
\item That is also our understanding (little or no benefit from manual builds), 
   and why I will personally be responsible for automating DevOps as much as possible.
 
\item We currently anticipate internal sprints. Our project governance arrangement
   includes `ad hoc user panels' who will advise about features and also complete UATs.
   We think it will be possible to involve them in pre-alpha testing of discrete 
   components of the system as they become minimally operational, offering an
   opportunity for early feedback. We did something similar for FAIMS 2 and it was
   very valuable. The user panels will certainly be involved in alpha testing.
   We will be open about progress and feedback, with a public road map (as per Rory 
   Macneil's recommendation, see below). 
 
\item Data interoperability is important to us, see discussion around `external
  round-trip' as one of our design goals for this version. We would love
  for an ELN to be one of our partners and to explore interoperability. In 
  terms of dividing data capture and management between pieces of software, we 
  are focusing heavily on data capture in field contexts, and assume that 
  researchers would use more appropriate tools for other data capture scenarios.
  We then want to provide a capacity for data export in a way that will make the 
  data as easy to reconcile (either by ingest into the other system, or into a third
  system for combining data). To address your example, artefact analysis is something that we've 
  considered a `core' activity that we'd do in FAIMS (since some of that analysis happens in 
  the field / offline). Other analysis that usually happens in a lab (say, palynological 
  analysis, radiometric dating, stable isotope analysis, etc.) would be better covered by an ELN
  built more for that purpose. If the environmental samples for such analysis were recorded
  using FAIMS, we could connect the data by (for example) assigning a persistent identifier (PID) in the 
  field (e.g., an IGSN), printing a bar or QR code with the PID to include with the sample, scanning 
  in that code in the lab, completing the analysis, and then reuniting the field and lab data based 
  on the PID. Or, if it were easier, we could export the field data for ingest into the ELN 
  as sample metadata. We're focusing on doing one thing - offline data capture - as well as possible, 
  and then federating with other systems to do other things. Relating back to your API question, 
  we are aiming for an API that 3rd-party apps can interact with, and would like to pursue 
  interoperability with an ELN to allow integration of field and lab data. Ideally this 
  interoperability would go both ways - it can be useful to have lab results when out 
  in the field on subsequent fieldwork. As Brian mentioned, we're trying something 
  analogous with Cloudstor (OwnCloud) interoperability (to allow online viewing / 
  editing of data in OpenRefine or a GIS).
 
\item Thanks! When we have APIs for user acceptance testing, we will make
  sure to let you know.
 
\end{enumerate}
\end{quote}
From Jeff Good
\begin{quote}
I read through the document and found it to be very well explained and
thorough. The choices seem well justified to me (though I can't speak to
direct knowledge of many of the technologies).
\end{quote}

From Nathan Reid
\begin{quote}
I have read through and am happy with the document. The flow makes sense
and I understand why certain options were chosen.
\end{quote}

From Jonathan Smillie:
\begin{quote}
 \begin{enumerate}[itemsep=1em]
\item `The code and platform should last for at least five years, assuming 
   regular maintenance'. {[}Is this an{]} expectation, or requirement?

\item Is `server' here a central server per-end-user deployment? Or a single 
   central server hosted by the FAIMS platform project itself? 
   (Or are app instances acting as distributed peer-to-peer servers?)

\item {[}Is{]} Export via app, or direct from server-side CouchDB instance?

\end{enumerate}
\end{quote}

Response to Jonathan Smillie:
\begin{quote}
 \begin{enumerate}[itemsep=1em]

\item Design expectation. A 5-year build time is based on our experience with FAIMS 2 and the 
   pace of change in mobile technology. We have geared our business plan towards this expectation.

\item The `server' represents a client's main datastore. Our hope is that each 
   instance of the app can act as a `server' (since that will mean that we only 
   need to code one app). While we will be offering hosted (online) `servers' for clients as part of our sustainability plan, 
   being able to deploy offline to local hardware is essential for off-the-grid research teams.

\item We have not yet specified export mechanism. The answer, we suspect, is `all of the above' -- 
   We are likely to use one or both of the approaches you suggest, plus our plugin architecture implies 
   that data-manipulation javascript can be loaded as part of the client. Partner organisations like AARNet
   and the data repositories will also be working with us on export capabilities. 

\end{enumerate}
\end{quote}
From Rory Macneil, email and conversation (excerpted / paraphrased):
\begin{quote}
\begin{enumerate}[itemsep=1em]

\item Ensure that development is market-led / customer-led rather than technically led by 
   involving users or clients at every stage of testing, getting feedback early and often 
   (and responding to it). 
   
\item UI/UX is everything (for uptake and ultimate success). Invest in it appropriately. UI/UX can serve as a bridge 
   between the development team and the commercial or client-facing side of the operation.
   Do detailed documentation as you develop each feature, involving the dev team. Publicise
   development early, providing a roadmap. Make short videos about new features. 
   Alongside involving users in testing, these activities
   can build a community of interested people who are potential customers. 
   
\item Investigate tools for customer engagement. we use Intercom (statistics on users and usage),
   Simpo (on-boarding tool for non-technical users), HelpDocs (documentation), noting that these tools 
   integrate with one another.
   
\item During development, RSpace (specifically referring to the Inventory Hub product) provides:
   \begin{itemize}
     \item Two-minute videos covering key features in each release.
     \item An easy-to-understand Roadmap (on Trello NOT github because our users are not developers).
     \item Full user documentation of all features as they are developed.
     \item Professionally designed and delivered user testing sessions.
     \item Sandbox instance where anyone can try out the latest version.
   \end{itemize}
\item Design your permissions system early; SSO integration is painful (but necessary for enterprises).

\end{enumerate}
\end{quote}

Response to Rory Macneil

\begin{quote}
Thank you! We have consciously attempted to ensure our development is market/customer-driven, 
but this reminder is timely. We have also used some of these approaches in the past, but 
not all of them, especially as part of a `package' like you suggest. We will work to 
incorporate these suggestions, although some will require additional resourcing 
(which we are currently seeking).
\end{quote}

%----------------------------------------------------------------------------------------

\end{document}
